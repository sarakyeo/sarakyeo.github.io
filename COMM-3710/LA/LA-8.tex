% Options for packages loaded elsewhere
\PassOptionsToPackage{unicode}{hyperref}
\PassOptionsToPackage{hyphens}{url}
\PassOptionsToPackage{dvipsnames,svgnames,x11names}{xcolor}
%
\documentclass[
]{article}

\usepackage{amsmath,amssymb}
\usepackage{iftex}
\ifPDFTeX
  \usepackage[T1]{fontenc}
  \usepackage[utf8]{inputenc}
  \usepackage{textcomp} % provide euro and other symbols
\else % if luatex or xetex
  \usepackage{unicode-math}
  \defaultfontfeatures{Scale=MatchLowercase}
  \defaultfontfeatures[\rmfamily]{Ligatures=TeX,Scale=1}
\fi
\usepackage{lmodern}
\ifPDFTeX\else  
    % xetex/luatex font selection
\fi
% Use upquote if available, for straight quotes in verbatim environments
\IfFileExists{upquote.sty}{\usepackage{upquote}}{}
\IfFileExists{microtype.sty}{% use microtype if available
  \usepackage[]{microtype}
  \UseMicrotypeSet[protrusion]{basicmath} % disable protrusion for tt fonts
}{}
\makeatletter
\@ifundefined{KOMAClassName}{% if non-KOMA class
  \IfFileExists{parskip.sty}{%
    \usepackage{parskip}
  }{% else
    \setlength{\parindent}{0pt}
    \setlength{\parskip}{6pt plus 2pt minus 1pt}}
}{% if KOMA class
  \KOMAoptions{parskip=half}}
\makeatother
\usepackage{xcolor}
\usepackage[top=20mm,left=20mm,right=20mm,heightrounded]{geometry}
\setlength{\emergencystretch}{3em} % prevent overfull lines
\setcounter{secnumdepth}{-\maxdimen} % remove section numbering
% Make \paragraph and \subparagraph free-standing
\ifx\paragraph\undefined\else
  \let\oldparagraph\paragraph
  \renewcommand{\paragraph}[1]{\oldparagraph{#1}\mbox{}}
\fi
\ifx\subparagraph\undefined\else
  \let\oldsubparagraph\subparagraph
  \renewcommand{\subparagraph}[1]{\oldsubparagraph{#1}\mbox{}}
\fi


\providecommand{\tightlist}{%
  \setlength{\itemsep}{0pt}\setlength{\parskip}{0pt}}\usepackage{longtable,booktabs,array}
\usepackage{calc} % for calculating minipage widths
% Correct order of tables after \paragraph or \subparagraph
\usepackage{etoolbox}
\makeatletter
\patchcmd\longtable{\par}{\if@noskipsec\mbox{}\fi\par}{}{}
\makeatother
% Allow footnotes in longtable head/foot
\IfFileExists{footnotehyper.sty}{\usepackage{footnotehyper}}{\usepackage{footnote}}
\makesavenoteenv{longtable}
\usepackage{graphicx}
\makeatletter
\def\maxwidth{\ifdim\Gin@nat@width>\linewidth\linewidth\else\Gin@nat@width\fi}
\def\maxheight{\ifdim\Gin@nat@height>\textheight\textheight\else\Gin@nat@height\fi}
\makeatother
% Scale images if necessary, so that they will not overflow the page
% margins by default, and it is still possible to overwrite the defaults
% using explicit options in \includegraphics[width, height, ...]{}
\setkeys{Gin}{width=\maxwidth,height=\maxheight,keepaspectratio}
% Set default figure placement to htbp
\makeatletter
\def\fps@figure{htbp}
\makeatother

\usepackage{booktabs}
\usepackage{longtable}
\usepackage{array}
\usepackage{multirow}
\usepackage{wrapfig}
\usepackage{float}
\usepackage{colortbl}
\usepackage{pdflscape}
\usepackage{tabu}
\usepackage{threeparttable}
\usepackage{threeparttablex}
\usepackage[normalem]{ulem}
\usepackage{makecell}
\usepackage{xcolor}
\makeatletter
\@ifpackageloaded{tcolorbox}{}{\usepackage[skins,breakable]{tcolorbox}}
\@ifpackageloaded{fontawesome5}{}{\usepackage{fontawesome5}}
\definecolor{quarto-callout-color}{HTML}{909090}
\definecolor{quarto-callout-note-color}{HTML}{0758E5}
\definecolor{quarto-callout-important-color}{HTML}{CC1914}
\definecolor{quarto-callout-warning-color}{HTML}{EB9113}
\definecolor{quarto-callout-tip-color}{HTML}{00A047}
\definecolor{quarto-callout-caution-color}{HTML}{FC5300}
\definecolor{quarto-callout-color-frame}{HTML}{acacac}
\definecolor{quarto-callout-note-color-frame}{HTML}{4582ec}
\definecolor{quarto-callout-important-color-frame}{HTML}{d9534f}
\definecolor{quarto-callout-warning-color-frame}{HTML}{f0ad4e}
\definecolor{quarto-callout-tip-color-frame}{HTML}{02b875}
\definecolor{quarto-callout-caution-color-frame}{HTML}{fd7e14}
\makeatother
\makeatletter
\makeatother
\makeatletter
\makeatother
\makeatletter
\@ifpackageloaded{caption}{}{\usepackage{caption}}
\AtBeginDocument{%
\ifdefined\contentsname
  \renewcommand*\contentsname{Table of contents}
\else
  \newcommand\contentsname{Table of contents}
\fi
\ifdefined\listfigurename
  \renewcommand*\listfigurename{List of Figures}
\else
  \newcommand\listfigurename{List of Figures}
\fi
\ifdefined\listtablename
  \renewcommand*\listtablename{List of Tables}
\else
  \newcommand\listtablename{List of Tables}
\fi
\ifdefined\figurename
  \renewcommand*\figurename{Figure}
\else
  \newcommand\figurename{Figure}
\fi
\ifdefined\tablename
  \renewcommand*\tablename{Table}
\else
  \newcommand\tablename{Table}
\fi
}
\@ifpackageloaded{float}{}{\usepackage{float}}
\floatstyle{ruled}
\@ifundefined{c@chapter}{\newfloat{codelisting}{h}{lop}}{\newfloat{codelisting}{h}{lop}[chapter]}
\floatname{codelisting}{Listing}
\newcommand*\listoflistings{\listof{codelisting}{List of Listings}}
\makeatother
\makeatletter
\@ifpackageloaded{caption}{}{\usepackage{caption}}
\@ifpackageloaded{subcaption}{}{\usepackage{subcaption}}
\makeatother
\makeatletter
\@ifpackageloaded{tcolorbox}{}{\usepackage[skins,breakable]{tcolorbox}}
\makeatother
\makeatletter
\@ifundefined{shadecolor}{\definecolor{shadecolor}{rgb}{.97, .97, .97}}
\makeatother
\makeatletter
\makeatother
\makeatletter
\makeatother
\ifLuaTeX
  \usepackage{selnolig}  % disable illegal ligatures
\fi
\IfFileExists{bookmark.sty}{\usepackage{bookmark}}{\usepackage{hyperref}}
\IfFileExists{xurl.sty}{\usepackage{xurl}}{} % add URL line breaks if available
\urlstyle{same} % disable monospaced font for URLs
\hypersetup{
  pdftitle={LA-8: Practice Data Analysis (15 points)},
  colorlinks=true,
  linkcolor={blue},
  filecolor={Maroon},
  citecolor={Blue},
  urlcolor={blue},
  pdfcreator={LaTeX via pandoc}}

\title{LA-8: Practice Data Analysis (15 points)}
\author{}
\date{}

\begin{document}
\maketitle
\ifdefined\Shaded\renewenvironment{Shaded}{\begin{tcolorbox}[breakable, boxrule=0pt, enhanced, sharp corners, borderline west={3pt}{0pt}{shadecolor}, frame hidden, interior hidden]}{\end{tcolorbox}}\fi

\hypertarget{learning-outcomes}{%
\section{Learning Outcomes}\label{learning-outcomes}}

In this assignment, you will practice the data analysis skills that you
have learned this semester.

\begin{tcolorbox}[enhanced jigsaw, opacityback=0, left=2mm, toptitle=1mm, bottomtitle=1mm, breakable, toprule=.15mm, opacitybacktitle=0.6, colback=white, bottomrule=.15mm, colframe=quarto-callout-tip-color-frame, coltitle=black, titlerule=0mm, title=\textcolor{quarto-callout-tip-color}{\faLightbulb}\hspace{0.5em}{Tip}, leftrule=.75mm, rightrule=.15mm, colbacktitle=quarto-callout-tip-color!10!white, arc=.35mm]

Read all the instructions carefully before starting the assignment.

\end{tcolorbox}

\begin{center}\rule{0.5\linewidth}{0.5pt}\end{center}

\hypertarget{instructions}{%
\section{Instructions}\label{instructions}}

\begin{enumerate}
\def\labelenumi{\arabic{enumi})}
\item
  Download the full COVID dataset (\texttt{covid-full.csv}) and codebook
  (\texttt{covid-full-codebook.csv}). Use the codebook to familiarize
  yourself with the variables in the data and read the data into R.
\item
  Create a new variable called \texttt{asia\_europe}. Have the variable
  equal \texttt{Asia} if the country is China, India, Japan, Singapore,
  or South Korea. Have the variable equal \texttt{Europe} is the country
  is Denmark, France, Germany, Italy, or Spain.\\
  \strut \\
  Obtain descriptive statistics for the number of time respondents
  washed their hands on the previous day for Asia and Europe.

  \begin{enumerate}
  \def\labelenumii{\alph{enumii})}
  \tightlist
  \item
    What is the mean and standard deviation of handwashes in Asia?
  \item
    What is the mean and standard deviation of handwashes in Europe?
  \end{enumerate}
\item
  Conduct a statistical test to determine whether there is a significant
  difference in the number of times people living in Asia compared to
  Europe washed their hands.

  \begin{enumerate}
  \def\labelenumii{\alph{enumii})}
  \tightlist
  \item
    State your hypothesis.
  \item
    Select and conduct a statistical test to address your hypothesis.
  \item
    Report the test statistic and \emph{p}-value.
  \item
    Does the statistical test support or refute your hypothesis?
  \end{enumerate}
\item
  Implement the appropriate statistical test to determine whether there
  is a significant linear relationship between how many times people
  left their home and the total number of contacts they had with other
  people.

  \begin{enumerate}
  \def\labelenumii{\alph{enumii})}
  \tightlist
  \item
    State your hypothesis.
  \item
    Select and conduct a statistical test to address your hypothesis.
  \item
    Report the test statistic and \emph{p}-value.
  \item
    Does the statistical test support or refute your hypothesis? If
    there is a significant relationship, describe it. Does this finding
    make sense? Why or why not?
  \end{enumerate}
\item
  Is there a difference in whether the majority of respondents in Asia
  compared to those in Europe said their lives were impacted by COVID?

  \begin{enumerate}
  \def\labelenumii{\alph{enumii})}
  \tightlist
  \item
    State your hypothesis.
  \item
    Select and conduct a statistical test to address your hypothesis.
  \item
    Report the test statistic and \emph{p}-value.
  \item
    Does the statistical test support or refute your hypothesis? If
    there is a significant difference, describe it (e.g., which region,
    Asia or Europe, had more respondents who said their lives were
    impacted by COVID). You can use the \texttt{filter()} and
    \texttt{freq()} functions to help you describe the differences.
  \end{enumerate}
\item
  \textbf{BONUS:} Use the data and a statistical test to determine
  whether there is a relationship between the number of days since the
  COVID outbreak began and how often respondents wore masks outside.
  Conduct the statistical test, report the results of the test (i.e.,
  test statistic and \emph{p}-value) and explain your conclusion.
\end{enumerate}

\begin{center}\rule{0.5\linewidth}{0.5pt}\end{center}

\hypertarget{submission}{%
\section{Submission}\label{submission}}

Submit your R script (named \texttt{LA-\#\_FirstName-LastName.R}) to
Canvas.

Your R script should:

\begin{enumerate}
\def\labelenumi{\arabic{enumi})}
\tightlist
\item
  Include commands and functions that are necessary to address all the
  questions in the assignment.
\item
  Contain comments that answer the questions in the assignment.
\item
  Run in its entirety without errors.
\end{enumerate}

To ensure that your R script runs without errors, you should:

\begin{itemize}
\tightlist
\item
  Save your script.
\item
  Navigate back to Your Workspace on Posit Cloud.
\item
  Reopen your project.
\item
  Run the entire script line-by-line without editing it to ensure there
  are no errors.
\end{itemize}

\begin{tcolorbox}[enhanced jigsaw, opacityback=0, left=2mm, toptitle=1mm, bottomtitle=1mm, breakable, toprule=.15mm, opacitybacktitle=0.6, colback=white, bottomrule=.15mm, colframe=quarto-callout-important-color-frame, coltitle=black, titlerule=0mm, title=\textcolor{quarto-callout-important-color}{\faExclamation}\hspace{0.5em}{Important}, leftrule=.75mm, rightrule=.15mm, colbacktitle=quarto-callout-important-color!10!white, arc=.35mm]

These standards apply to all submissions in this course that require R
scripts. You should follow these instructions for preparation, naming,
and saving of your R script for all of your individual lab assignments.

\end{tcolorbox}



\end{document}
