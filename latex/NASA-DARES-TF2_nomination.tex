% main.tex - simple LaTeX template
% \documentclass[12pt]{article}

% % Packages
% \usepackage[T1]{fontenc}
% \usepackage[utf8]{inputenc}
% \usepackage{lmodern}
% \usepackage{microtype}
% \usepackage{amsmath,amssymb}
% \usepackage{graphicx}
% \usepackage{booktabs}
% \usepackage[letterpaper, margin=1in]{geometry}
% \usepackage{hyperref}
% \usepackage{enumitem}
% \usepackage{csquotes}
% \usepackage{caption}
% \usepackage{float}
% \usepackage{tikz}

% \usepackage[style=apa, backend=biber]{biblatex}
% \addbibresource{refs.bib}

\PassOptionsToPackage{quiet}{fontspec}
\documentclass[12pt]{letter}
\usepackage{ctex}
\usepackage[utf8]{inputenc}
\usepackage[T1]{fontenc}
\usepackage{amsmath}
\usepackage{amssymb}
\usepackage{graphicx}
\usepackage[left=1in, right=1in, top=1in, bottom=1in]{geometry}
\usepackage{times} %Times New Roman font
\usepackage{tabularx}
\usepackage{setspace}
\usepackage{color}
\usepackage{multirow}
\usepackage{titlesec}
    \titleformat{\section}{\normalfont\bfseries}{\thesection}{1em}{}
\usepackage{csquotes}
\usepackage[nodayofweek]{datetime}
       
\usepackage{hyperref}
    \hypersetup{
        colorlinks=true,
        linkcolor=blue,
        urlcolor=blue,
        filecolor=blue,
        citecolor=black}

\usepackage[style=apa, backend=biber]{biblatex}
\addbibresource{refs.bib}
    
\setlength{\parindent}{0pt} %Paragraph indentation
\setlength{\parskip}{1em} %Vertical space between paragraphs

% Metadata
% \title{Title of the Document}
% \author{Author Name}
% \date{\today}

\begin{document}

%-----------------------------Letterhead-----------------------------%
% \hfill
\begin{minipage}{4in}
    \vspace{-4em}
    \begin{centering}
        \includegraphics[width=6.5in]{Communication-letterhead.png}
    \end{centering}
    \vspace{0.5em}
\end{minipage}
% \hfill %Horizontal fill 

\vspace{-1.5em}% vertical space
% \rule{\linewidth}{1pt} % Horizontal rule
% \vspace{-2.5em}% make the date more close to the horizontal line
%-----------------------------text-----------------------------%

% \textbf{Subject here}\\
% \rule{\linewidth}{0.5pt}

I am writing to express my strong interest in serving as a NASA Decadal Astrobiology Research and Exploration Strategy (DARES) Task Force 2 General Member or Focus Area Lead. My background, expertise, and leadership in science communication align well with Focus Area 9 (Astrobiology in Society) and are directly relevant to Topic 8 (Prepare for the Discovery of Life Beyond Earth and Subsequent Post-Discovery Activities) of NASA-DARES. I am interested in continuing conversations and work that I began as part of the Scientific Organizing Committee (SOC) for the 2024 NASA workshop, Communicating Discoveries in the Search for Life in the Universe (CDSLU). The societal implications of any discovery of life beyond Earth are profound and far-reaching, and will have significance for all aspects of humanity. It is thus imperative that we are prepared to responsibly and ethically communicate about astrobiology research that finds evidence of life beyond Earth. In this endeavor, it is critical that we rely on empirical evidence from science communication.

I have a multifaceted background in science and communication, which has informed my research program. My scholarship is centered on understanding the processes through which public audiences form attitudes, opinions, and beliefs, and make decisions in the context of media representations of and messages about scientific issues. Broadly, my work focuses on strategic tactics for communicating information in the context of diverse scientific issues, including climate change, renewable energy, biomanufacturing, and astrobiology. Of relevance to TF2, I conducted a study on the Arsenic bacteria controversy \parencite{yeoCaseArseniclifeBlogs2017}, which exeplifies the crucial role of media in post-publication peer review, and the need to understand and be strategic about communication at the intersection of science, media, and society.

Throughout my career, I have contributed to national and international public science communication and engagement efforts that bridge research and society. In addition to my service on the CDSLU SOC, I co-led a section of the CDSLU workshop report \parencite{bimmWorkshopReportCommunicatingforthcoming}. I have also served on the Steering Committee of the Science and Public Engagement Partnership (SciPEP) between the Kavli Foundation and the Department of Energy's Office of Science. I am currently on the Standing Committee for Advancing Science Communication and the Board on Life Sciences in the National Academies of Sciences, Engineering, and Medicine (NASEM). Internationally, I have served on the Scientific Committee of the Global Network for Science Communication (PCST) and the Scientific Advisory Committee for the International Symposium on Communicating Discovery Science, which was held in Stellenbosch, South Africa, in 2024.

In addition to my expertise and service experience, I am the Director of the STEM Ambassador Program (STEMAP)\footnote{\href{https://stemap.utah.edu/}{https://stemap.utah.edu/}}, a public engagement training program that equips scientists with practical skills and knowledge to effectively connect with audiences that are typically underserved by traditional modes of science communication. This role, combined with my experience in leading research projects in science communication, participating in conference and workshop planning, and serving in advisory roles, has provided me with a broad perspective of the landscape of societal impacts of research and a nuanced understanding of the challenges and opportunities that scientists face when engaging with public audiences across diverse contexts.

I am excited about the opportunity to collaborate with other experts to advance NASA Astrobiology's strategic goals and am open to serving as a Focus Area Lead. My goals are two-fold: First, I hope to advance our understanding and the role of communication in astrobiology by applying lessons learned from empirical scholarship in science communication. Second, I hope to encourage more scholarship related to astrobiology in society among my colleagues and peers in science communication and related disciplines. My expertise, experiences, and leadership skills position me well to contribute meaningfully to TF2's work. I am committed to fostering an collaborative environment that values multiple perspectives and expertise. I am confident that my background, expertise, skills, and dedication to science communication will enable me to make significant contributions to the NASA-DARES Task Force 2.

Thank you for considering this nomination. I look forward to hearing from you and the opportunity to contribute to this important work.

\vspace{1em}

Sincerely,\\
\includegraphics[height=5em]{signature.png}\\
Sara K. Yeo\\
Professor\\
Department of Communication\\
University of Utah\\
\href{mailto:sara.yeo@utah.edu}{sara.yeo$@$utah.edu}

\newpage

\printbibliography
\end{document}