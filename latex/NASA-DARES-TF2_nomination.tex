% main.tex - simple LaTeX template
\documentclass[12pt]{article}

% Packages
\usepackage[T1]{fontenc}
\usepackage[utf8]{inputenc}
\usepackage{lmodern}
\usepackage{microtype}
\usepackage{amsmath,amssymb}
\usepackage{graphicx}
\usepackage{booktabs}
\usepackage[letterpaper, margin=1in]{geometry}
\usepackage{hyperref}
\usepackage{enumitem}
\usepackage{csquotes}
\usepackage{caption}
\usepackage{float}
\usepackage{tikz}

\usepackage[style=apa, backend=biber]{biblatex}
\addbibresource{refs.bib}

% Metadata
% \title{Title of the Document}
% \author{Author Name}
% \date{\today}


\begin{document}
% \maketitle
% 
% \begin{abstract}
% Write an abstract here.
% \end{abstract}

A two-page (maximum) cover letter indicating interest in and qualifications for serving on NASA-DARES TF2, including relevant scientific, technical, and management experience as it relates to the nine Focus Areas (\href{https://assets.science.nasa.gov/content/dam/science/psd/astrobiology/for-researchers/dares/9.%20Astrobiology%20in%20Society.pdf}{Focus Area 9: Astrobiology in Society}) or similar strategic planning activities. 

The cover letter may also describe prior contributions to NASA, the National Academies of Sciences, Engineering, and Medicine, or other community-based strategic planning efforts. Applicants who are willing to serve as a Co-Chair or Focus Area Lead should clearly state their interest and provide a rationale for why they are well suited for the role, referencing specific expertise, leadership skills, and relevant experience.

Maximum document length is 2 pages with a font size 12, not to exceed 15 characters per horizontal inch, including spaces, sans serif font recommended. Section page limits are not transferrable. Testing a citation here \parencite{yeoHumorCanIncrease2022}.

I am writing to express my strong interest in serving as a NASA Decadal Astrobiology Research and Exploration Strategy (DARES) Task Force 2 member. My background, expertise, and leadership in science communication aligns well with Focus Area 9 (Astrobiology in Society) of NASA-DARES.

% Throughout my career, I have contributed to national and international science communication and public engagement efforts that bridge scientific research and societal impact. My experience includes leading research projects on public understanding of science, participating in strategic planning workshops, and serving in advisory roles for organizations such as NASA and the National Academies. These roles have provided me with a comprehensive perspective on the challenges and opportunities in communicating complex scientific discoveries, especially those related to astrobiology.

% I am enthusiastic about the opportunity to collaborate with fellow experts to advance NASA’s strategic goals and am willing to serve as a Co-Chair or Focus Area Lead. My leadership skills, demonstrated through directing large-scale initiatives and facilitating interdisciplinary dialogue, position me well to contribute meaningfully to the task force’s work.

\begin{itemize}
    \item Interest in continuing the conversations and work that I began as part of the CDSLU Scientific Organizing Committee (also section co-lead).\item My expertise is directly relevant to DARES Topic 8: “Topic 8: Prepare for the Discovery of Life Beyond Earth and Subsequent Post-Discovery Activities,” and Focus Area 9: “Astrobiology and Society.”
    \item Qualifications: Science communication researcher and expert. Add expertise and research interests. Reference \textit{PUS} paper on \#arseniclife.

    \item Participation on Planning Committee of CDSLU Workshop held in 2024 (check date).
    \item Contributed to the CDSLU ``Workshop Report: Communicating Discoveries in the Search for Life in the Universe.'' Co-led the writing and organization of the section on ``Surprises, Disconnects, Misunderstandings'' \parencite{bimmWorkshopReportCommunicatingforthcoming}.
    \item STEMAP Director position as one that is focused on societal impacts of research--offers a bird's eye view of the societal impacts landscape.
    \item Served on Steering Committee for SciPEP.
    \item Serve on Standing Committee for Advancing Science Communication in the National Academies.
    \item Serve on the Board on Life Sciences in NASEM.
\end{itemize}

\newpage

\printbibliography
\end{document}