% main.tex - simple LaTeX template
\documentclass[12pt]{article}

% Packages
\usepackage[T1]{fontenc}
\usepackage[utf8]{inputenc}
\usepackage{lmodern}
\usepackage{microtype}
\usepackage{amsmath,amssymb}
\usepackage{graphicx}
\usepackage{booktabs}
\usepackage[letterpaper, margin=1in]{geometry}
\usepackage{hyperref}
\usepackage{enumitem}
\usepackage{csquotes}
\usepackage{caption}
\usepackage{float}
\usepackage{tikz}

\usepackage[style=apa, backend=biber]{biblatex}
\addbibresource{refs.bib}

% Metadata
% \title{Title of the Document}
% \author{Author Name}
% \date{\today}


\begin{document}
% \maketitle
% 
% \begin{abstract}
% Write an abstract here.
% \end{abstract}

I am writing to express my strong interest in serving as a NASA Decadal Astrobiology Research and Exploration Strategy (DARES) Task Force 2 member. My background, expertise, and leadership in science communication aligns well with Focus Area 9 (Astrobiology in Society) and are directly relevant to NASA-DARES Topic 8 (Prepare for the Discovery of Life Beyond Earth and Subsequent Post-Discovery Activities). I am interested in continuing and contributing to conversations and work that I began as part of the Scientific Organizing Committee for the 2024 NASA workshop, Communicating Discoveries in the Search for Life in the Universe. The societal implications of any discovery of life beyond Earth are profound and far-reaching, unbounded by national borders, and will have significance for all aspects of humanity. It is thus imperative that we are prepared to responsibly and ethically communicate about astrobiology research that finds evidence of life beyond Earth. In this endeavor, it is equally critical that we rely on empirical evidence from science communication, not on intuition.

\vspace{1em}

I have a multifaceted background in science and communication. I hold a BS in Oceanography, a MS in Ocenography and another in Life Sciences Communication, and a PhD in Science Communication. My circuitous journey to the discipline of Communication is central to my scholarship. My research is centered on understanding the processes through which public audiences form attitudes, opinions, and make decisions in the context of media representations of and messages about scientific issues. Broadly, my work focuses on strategic tactics for communicating information in the context of diverse issues in science, including climate change, renewable energy, biomanufacturing, and astrobiology. Notably, I conducted a study on the Arsenic bacteria controversy \parencite{yeoCaseArseniclifeBlogs2017} in which I used both qualitative and quantitative methods to analyze how public audiences engaged with and interpreted the scientific claims made in the retracted \textit{Science} article \parencite{wolfe-simonBacteriumThatCan2011}. This case study exeplifies the crucial role of media in post-publication peer review and the need to understand the intersection of science, media, and society.

\vspace{1em}

Throughout my career, I have contributed to national and international science communication and public engagement efforts that bridge scientific research and societal impact. At the local level, I serve as an Advisory Board Member for the graduate student-led group, Utah Science Communicators (USCo).

My experience includes leading research projects on public understanding of science, participating in strategic planning workshops, and serving in advisory roles for organizations such as NASA and the National Academies. These roles have provided me with a comprehensive perspective on the challenges and opportunities in communicating complex scientific discoveries, especially those related to astrobiology.

% I am enthusiastic about the opportunity to collaborate with fellow experts to advance NASA’s strategic goals and am open to serving as a Co-Chair or Focus Area Lead. My leadership skills, demonstrated through directing large-scale initiatives and facilitating interdisciplinary dialogue, position me well to contribute meaningfully to TF2’s work.

\begin{itemize}
    \item Interest in continuing the conversations and work that I began as part of the CDSLU Scientific Organizing Committee (also section co-lead).
    \item My expertise is directly relevant to DARES Topic 8: “Topic 8: Prepare for the Discovery of Life Beyond Earth and Subsequent Post-Discovery Activities,” and Focus Area 9: “Astrobiology and Society.”
    \item Qualifications: Science communication researcher and expert. Add expertise and research interests. Reference \textit{PUS} paper on \#arseniclife.
    \item Participation on Planning Committee of CDSLU Workshop held in 2024 (check date).
    \item Contributed to the CDSLU ``Workshop Report: Communicating Discoveries in the Search for Life in the Universe.'' Co-led the writing and organization of the section on ``Surprises, Disconnects, Misunderstandings'' \parencite{bimmWorkshopReportCommunicatingforthcoming}.
    \item STEMAP Director position as one that is focused on societal impacts of research--offers a bird's eye view of the societal impacts landscape.
    \item Served on Steering Committee for SciPEP.
    \item Serve on Standing Committee for Advancing Science Communication in the National Academies.
    \item Serve on the Board on Life Sciences in NASEM.
\end{itemize}

\newpage

\printbibliography
\end{document}