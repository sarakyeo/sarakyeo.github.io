% Options for packages loaded elsewhere
\PassOptionsToPackage{unicode}{hyperref}
\PassOptionsToPackage{hyphens}{url}
\PassOptionsToPackage{dvipsnames,svgnames,x11names}{xcolor}
%
\documentclass[
  letterpaper,
]{article}

\usepackage{amsmath,amssymb}
\usepackage{iftex}
\ifPDFTeX
  \usepackage[T1]{fontenc}
  \usepackage[utf8]{inputenc}
  \usepackage{textcomp} % provide euro and other symbols
\else % if luatex or xetex
  \usepackage{unicode-math}
  \defaultfontfeatures{Scale=MatchLowercase}
  \defaultfontfeatures[\rmfamily]{Ligatures=TeX,Scale=1}
\fi
\usepackage{lmodern}
\ifPDFTeX\else  
    % xetex/luatex font selection
\fi
% Use upquote if available, for straight quotes in verbatim environments
\IfFileExists{upquote.sty}{\usepackage{upquote}}{}
\IfFileExists{microtype.sty}{% use microtype if available
  \usepackage[]{microtype}
  \UseMicrotypeSet[protrusion]{basicmath} % disable protrusion for tt fonts
}{}
\makeatletter
\@ifundefined{KOMAClassName}{% if non-KOMA class
  \IfFileExists{parskip.sty}{%
    \usepackage{parskip}
  }{% else
    \setlength{\parindent}{0pt}
    \setlength{\parskip}{6pt plus 2pt minus 1pt}}
}{% if KOMA class
  \KOMAoptions{parskip=half}}
\makeatother
\usepackage{xcolor}
\usepackage[top=20mm,bottom=25mm,left=20mm,right=20mm,heightrounded]{geometry}
\setlength{\emergencystretch}{3em} % prevent overfull lines
\setcounter{secnumdepth}{-\maxdimen} % remove section numbering
% Make \paragraph and \subparagraph free-standing
\ifx\paragraph\undefined\else
  \let\oldparagraph\paragraph
  \renewcommand{\paragraph}[1]{\oldparagraph{#1}\mbox{}}
\fi
\ifx\subparagraph\undefined\else
  \let\oldsubparagraph\subparagraph
  \renewcommand{\subparagraph}[1]{\oldsubparagraph{#1}\mbox{}}
\fi


\providecommand{\tightlist}{%
  \setlength{\itemsep}{0pt}\setlength{\parskip}{0pt}}\usepackage{longtable,booktabs,array}
\usepackage{calc} % for calculating minipage widths
% Correct order of tables after \paragraph or \subparagraph
\usepackage{etoolbox}
\makeatletter
\patchcmd\longtable{\par}{\if@noskipsec\mbox{}\fi\par}{}{}
\makeatother
% Allow footnotes in longtable head/foot
\IfFileExists{footnotehyper.sty}{\usepackage{footnotehyper}}{\usepackage{footnote}}
\makesavenoteenv{longtable}
\usepackage{graphicx}
\makeatletter
\def\maxwidth{\ifdim\Gin@nat@width>\linewidth\linewidth\else\Gin@nat@width\fi}
\def\maxheight{\ifdim\Gin@nat@height>\textheight\textheight\else\Gin@nat@height\fi}
\makeatother
% Scale images if necessary, so that they will not overflow the page
% margins by default, and it is still possible to overwrite the defaults
% using explicit options in \includegraphics[width, height, ...]{}
\setkeys{Gin}{width=\maxwidth,height=\maxheight,keepaspectratio}
% Set default figure placement to htbp
\makeatletter
\def\fps@figure{htbp}
\makeatother

\makeatletter
\@ifpackageloaded{tcolorbox}{}{\usepackage[skins,breakable]{tcolorbox}}
\@ifpackageloaded{fontawesome5}{}{\usepackage{fontawesome5}}
\definecolor{quarto-callout-color}{HTML}{909090}
\definecolor{quarto-callout-note-color}{HTML}{0758E5}
\definecolor{quarto-callout-important-color}{HTML}{CC1914}
\definecolor{quarto-callout-warning-color}{HTML}{EB9113}
\definecolor{quarto-callout-tip-color}{HTML}{00A047}
\definecolor{quarto-callout-caution-color}{HTML}{FC5300}
\definecolor{quarto-callout-color-frame}{HTML}{acacac}
\definecolor{quarto-callout-note-color-frame}{HTML}{4582ec}
\definecolor{quarto-callout-important-color-frame}{HTML}{d9534f}
\definecolor{quarto-callout-warning-color-frame}{HTML}{f0ad4e}
\definecolor{quarto-callout-tip-color-frame}{HTML}{02b875}
\definecolor{quarto-callout-caution-color-frame}{HTML}{fd7e14}
\makeatother
\makeatletter
\makeatother
\makeatletter
\makeatother
\makeatletter
\@ifpackageloaded{caption}{}{\usepackage{caption}}
\AtBeginDocument{%
\ifdefined\contentsname
  \renewcommand*\contentsname{Table of contents}
\else
  \newcommand\contentsname{Table of contents}
\fi
\ifdefined\listfigurename
  \renewcommand*\listfigurename{List of Figures}
\else
  \newcommand\listfigurename{List of Figures}
\fi
\ifdefined\listtablename
  \renewcommand*\listtablename{List of Tables}
\else
  \newcommand\listtablename{List of Tables}
\fi
\ifdefined\figurename
  \renewcommand*\figurename{Figure}
\else
  \newcommand\figurename{Figure}
\fi
\ifdefined\tablename
  \renewcommand*\tablename{Table}
\else
  \newcommand\tablename{Table}
\fi
}
\@ifpackageloaded{float}{}{\usepackage{float}}
\floatstyle{ruled}
\@ifundefined{c@chapter}{\newfloat{codelisting}{h}{lop}}{\newfloat{codelisting}{h}{lop}[chapter]}
\floatname{codelisting}{Listing}
\newcommand*\listoflistings{\listof{codelisting}{List of Listings}}
\makeatother
\makeatletter
\@ifpackageloaded{caption}{}{\usepackage{caption}}
\@ifpackageloaded{subcaption}{}{\usepackage{subcaption}}
\makeatother
\makeatletter
\@ifpackageloaded{tcolorbox}{}{\usepackage[skins,breakable]{tcolorbox}}
\makeatother
\makeatletter
\@ifundefined{shadecolor}{\definecolor{shadecolor}{rgb}{.97, .97, .97}}
\makeatother
\makeatletter
\makeatother
\makeatletter
\makeatother
\ifLuaTeX
  \usepackage{selnolig}  % disable illegal ligatures
\fi
\IfFileExists{bookmark.sty}{\usepackage{bookmark}}{\usepackage{hyperref}}
\IfFileExists{xurl.sty}{\usepackage{xurl}}{} % add URL line breaks if available
\urlstyle{same} % disable monospaced font for URLs
\hypersetup{
  colorlinks=true,
  linkcolor={blue},
  filecolor={Maroon},
  citecolor={Blue},
  urlcolor={blue},
  pdfcreator={LaTeX via pandoc}}

\title{COMM 5117: Health Campaigns and Media}
\usepackage{etoolbox}
\makeatletter
\providecommand{\subtitle}[1]{% add subtitle to \maketitle
  \apptocmd{\@title}{\par {\large #1 \par}}{}{}
}
\makeatother
\subtitle{Section 301\\
\strut \\
Mondays - Fridays (9:00 am - 12:00 pm)\\
Location: University of Utah Asia Campus\\
\strut \\
Professor: Dr.~Sara K. Yeo\\
Email: \href{mailto:sara.yeo@utah.edu}{\nolinkurl{sara.yeo@utah.edu}}}
\author{}
\date{}

\begin{document}
\maketitle
\ifdefined\Shaded\renewenvironment{Shaded}{\begin{tcolorbox}[sharp corners, frame hidden, enhanced, borderline west={3pt}{0pt}{shadecolor}, boxrule=0pt, interior hidden, breakable]}{\end{tcolorbox}}\fi

\hypertarget{sec-outline}{%
\section{Course Outline}\label{sec-outline}}

This three-credit course provides an introduction to the application and
integration of media effects theories in health and risk communication.
We will examine social and cognitive models relevant to the context of
health and risk communication campaigns and address theoretical
perspectives that inform campaign messaging, including social
determinants of health, individual behavior change, information
processing, and message effects. Through this course, students will gain
an understanding of theories relevant to the study of health and risk
communication messages and an appreciation of the importance of
integrating theory in understanding persuasive message effects.

\begin{tcolorbox}[enhanced jigsaw, bottomrule=.15mm, leftrule=.75mm, toptitle=1mm, titlerule=0mm, breakable, toprule=.15mm, opacityback=0, colbacktitle=quarto-callout-note-color!10!white, colback=white, title=\textcolor{quarto-callout-note-color}{\faInfo}\hspace{0.5em}{Note}, bottomtitle=1mm, arc=.35mm, rightrule=.15mm, coltitle=black, colframe=quarto-callout-note-color-frame, left=2mm, opacitybacktitle=0.6]

You are expected to log into the course Canvas website regularly
(\textbf{at least 3-5 times per week}), complete and submit work on
time, and ask questions if you need help. \textbf{It is your
responsibility as a student to ask questions in a timely manner during
scheduled labs and office hours, if you need help.}

\end{tcolorbox}

\hypertarget{sec-text}{%
\section{Required Text and Readings}\label{sec-text}}

There is no single textbook or edited volume that adequately captures
the breadth and depth of this evolving area of research. Therefore, I
have compiled readings for each week that will be available as PDF files
unless they are directly available online. There are several books that
are relevant to this topic area. They are not required for this course,
but may serve as good references.

\begin{itemize}
\item
  Rice, R. E., \& Atkin, C. K. (2013). Public Communication Campaigns
  (R. E. Rice, Ed.; 4th ed.). Sage.
\item
  du Pré, A. (2013). Communicating About Health: Current Issues and
  Perspectives (4th ed.). Oxford University Press.
\end{itemize}

\hypertarget{sec-tech}{%
\section{Technology Requirements}\label{sec-tech}}

To ensure that you have full access to the course, you will need:

\begin{itemize}
\item
  Reliable access to a laptop or desktop computer. A mobile device
  (tablet, phone) is not sufficient to complete this course. Please
  bring a laptop to lab.
\item
  An Internet browser compatible with
  \href{https://utah.instructure.com/}{Canvas}. For more information,
  see this
  \href{https://community.canvaslms.com/docs/DOC-10720-67952720329}{page}.
  Announcements, assignments, readings, etc., will be posted there. You
  should be familiar with Canvas. If you need help with Canvas, visit
  the \href{https://community.canvaslms.com/docs/DOC-10701}{Canvas
  Getting Started Guide for Students}.
\end{itemize}

\hypertarget{sec-requirements}{%
\section{Course Requirements}\label{sec-requirements}}

Course grades will be based on the following:

\begin{itemize}
\tightlist
\item
  Attendance, participation, and discussion (20\%)
\item
  Response papers (20\%)
\item
  Case study: Presentation (15\%\%)
\item
  Case study: Written analysis (15\%)
\item
  Group Project: Campaign planning proposal (30\%)
\end{itemize}

\hypertarget{attendance-participation-and-discussion-20}{%
\subsection{Attendance, Participation, and Discussion
(20\%)}\label{attendance-participation-and-discussion-20}}

Much of the learning that occurs in this course will be in a
seminar-style, discussion setting. To make this work---and make this
fun---you will need to be prepared to vigorously debate and discuss the
material. It is not enough that you just come to class. You are expected
to actively discuss the readings and critically analyze their contents.
More information on the University attendance policy can be found
\href{https://catalog.utah.edu/pages/8Fjihzb4XouQv5OY0ZiE}{here}.

Our goal is \textbf{knowledge integration}--connecting seemingly
disparate ideas and fitting them together in a larger picture--to
provide a broad context for advancing our understanding. The best way to
integrate knowledge from this class with what you already know is to:

\begin{itemize}
\tightlist
\item
  Make sure you \textbf{read deeply}, actively drawing out the
  implications of the readings and connecting them to other concepts and
  ideas that you have learned.
\item
  \textbf{Participate actively} in the class, challenging the evidence
  provided by the studies, by me, or by other students.
\item
  \textbf{Ask questions} if there is something you do not understand.
  Other students are likely to have the same question.
\end{itemize}

\begin{tcolorbox}[enhanced jigsaw, bottomrule=.15mm, leftrule=.75mm, toptitle=1mm, titlerule=0mm, breakable, toprule=.15mm, opacityback=0, colbacktitle=quarto-callout-note-color!10!white, colback=white, title=\textcolor{quarto-callout-note-color}{\faInfo}\hspace{0.5em}{Note}, bottomtitle=1mm, arc=.35mm, rightrule=.15mm, coltitle=black, colframe=quarto-callout-note-color-frame, left=2mm, opacitybacktitle=0.6]

Respectful disagreement and healthy debate is good and encouraged in all
my courses.

\end{tcolorbox}

\hypertarget{response-papers-20}{%
\subsection{Response Papers (20\%)}\label{response-papers-20}}

You will write 3 response papers during the semester. Response papers
should be at least 1 page (single-spaced, 12-point font, 1-in margins).
A good response paper will summarize and critique the readings for the
day. Your response should identify common threads between the readings
and what you already know, and critically engage with the ideas and
concepts in the assigned readings.

Your response papers are due at the end of the day via Canvas for the
class for which you choose to write a response.

\hypertarget{case-study-presentations-15}{%
\subsection{Case Study: Presentations
(15\%)}\label{case-study-presentations-15}}

Over the course, each student will present one campaign. Presentations
will be scheduled during the first day of the course.

Presentations should include \textbf{(1) a summary of the case},
\textbf{(2) an analysis and a critique of the case}, and \textbf{(3)
your conclusions}. In the summary, you should provide enough information
about the case, including what the case was about, what the organization
did, the results of the campaign, etc. In the analysis and critique, you
must analyze the effectiveness of the campaign. To assess
effectiveness,you can compare your case to other case(s) and/or apply
any theories they have learned in this or other classes. In the
conclusion, you must determine whether the case serves as a good
benchmark of an excellent campaign.

Presentations are expected to take about \textbf{30 minutes} followed by
a Q\&A session.

While scheduling presentations later in the semester will allow students
more time to prepare, the expectation is that these presentations will
be of higher quality.

Please note that students who do not present their case study on the
assigned day will receive a 50\% deduction to their presentation grade
as a penalty. If any emergency occurs, please contact the instructor
ASAP.

After the presentation, you must submit the presentation slides on
Canvas.

\hypertarget{case-study-written-analyses-15}{%
\subsection{Case Study: Written Analyses
(15\%)}\label{case-study-written-analyses-15}}

Over the course, you will see other students' campaign case study
presentations. Each student will write and analysis of 10 campaigns. The
paper should include \textbf{(1) a short summary of the case} and
\textbf{(2) your own evaluation of the case (not the presenter's
evaluation)}, including what was good and what was not, and what they
would do if they were in charge of that campaign.

Each analysis should be \textbf{at least 250 words}. Due dates will be
posted on Canvas.

\hypertarget{group-project-campaign-planning-proposal-30}{%
\subsection{Group Project: Campaign Planning Proposal
(30\%)}\label{group-project-campaign-planning-proposal-30}}

Your project will be a campaign planning proposal where your team will
develop a health campaign plan related to a topic of your choice. The
proposal is designed to provide an opportunity for students to apply the
materials covered in readings and in class. Students are assigned to
groups and will collaborate to select a specific health or risky
behavior. More detailed information will be provided in class.

There will be assignments throughout the semester designed to help you
complete your proposal.

\begin{enumerate}
\def\labelenumi{\arabic{enumi})}
\item
  \textbf{Preliminary research presentation.} Teams will decide on a
  specific health or risky behavior to work on and will conduct
  preliminary research. Then, your team will present this research in
  class and receive feedback from other students and the instructor.
\item
  \textbf{Focus group interviews (FGI) protocol, transcription, and
  summary.} Based on your preliminary research, your team will identify
  your research needs and conduct focus group interviews (FGI) with
  other students in the class. Before conducting interviews, you will
  work on the FGI protocol for submission. After the interviews, your
  team will transcribe and summarize what you learned from the FGI.
\item
  \textbf{Final presentation.} Your team will present your health/risk
  communication campaign proposal in class.
\item
  \textbf{Written proposal.} Your team will prepare and submit a written
  proposal via Canvas at the end of the semester.
\end{enumerate}

\hypertarget{course-grading}{%
\section{Course Grading}\label{course-grading}}

Grades in this course will be based on the following scale.

\begin{longtable}[]{@{}lc@{}}
\toprule\noalign{}
Grade & Score (\%) \\
\midrule\noalign{}
\endhead
\bottomrule\noalign{}
\endlastfoot
A & 93 to 100 \\
A- & 90 to \textless{} 93 \\
B+ & 87 to \textless{} 90 \\
B & 83 to \textless{} 87 \\
B- & 80 to \textless{} 83 \\
C+ & 77 to \textless{} 80 \\
C & 73 to \textless{} 77 \\
C- & 70 to \textless{} 73 \\
D+ & 67 to \textless{} 70 \\
D & 63 to \textless{} 67 \\
D- & 60 to \textless{} 63 \\
E & \textless{} 60 \\
\end{longtable}

You can and should check your grade regularly on Canvas. Information on
the grade points assigned to letter grades and how to calculate your GPA
can be found
\href{https://advising.utah.edu/academic-standards/gpa-calculator-new.php}{here}.

\begin{tcolorbox}[enhanced jigsaw, bottomrule=.15mm, leftrule=.75mm, toptitle=1mm, titlerule=0mm, breakable, toprule=.15mm, opacityback=0, colbacktitle=quarto-callout-important-color!10!white, colback=white, title=\textcolor{quarto-callout-important-color}{\faExclamation}\hspace{0.5em}{Important}, bottomtitle=1mm, arc=.35mm, rightrule=.15mm, coltitle=black, colframe=quarto-callout-important-color-frame, left=2mm, opacitybacktitle=0.6]

If you wish to dispute your grade on any assignment, you must put your
concerns in writing (please adhere to the
\protect\hyperlink{sec-policies}{course email policy}) via email to
\href{sara.yeo@utah.edu}{Prof.~Yeo}, clearly outlining your rationale.
These concerns must be presented within one week of receiving your
grade.

\end{tcolorbox}

\hypertarget{sec-policies}{%
\section{Course Policies}\label{sec-policies}}

By enrolling in this course, you agree to:

\begin{enumerate}
\def\labelenumi{\arabic{enumi}.}
\tightlist
\item
  respect the instructor and all members of the course;
\item
  engage with the content meaningfully;
\item
  meet the requirements of this course; and
\item
  abide by the course policies outlined in the syllabus.
\end{enumerate}

This list represents the \textbf{minimal standards} to make the course a
productive learning space. \textbf{Your final grade may be reduced by
1\% each time you engage in disruptive and/or disrespectful behaviors}.

\hypertarget{email-policy}{%
\subsection{Email Policy}\label{email-policy}}

\begin{tcolorbox}[enhanced jigsaw, bottomrule=.15mm, leftrule=.75mm, toptitle=1mm, titlerule=0mm, breakable, toprule=.15mm, opacityback=0, colbacktitle=quarto-callout-note-color!10!white, colback=white, title=\textcolor{quarto-callout-note-color}{\faInfo}\hspace{0.5em}{Note}, bottomtitle=1mm, arc=.35mm, rightrule=.15mm, coltitle=black, colframe=quarto-callout-note-color-frame, left=2mm, opacitybacktitle=0.6]

It is critical that you check your University email account frequently
and that you use your University email account to contact your
instructors.

\end{tcolorbox}

\textbf{I will not respond to emails originating from a non-University
account (e.g., Google, Yahoo, etc.)}. Using a non-University account
runs the risk of your message being diverted to Spam/Junks and your
message may not reach me in a timely fashion, if at all. Emails should
be written clearly and professionally with correct spelling and grammar.
\textbf{Emails that do not conform to these rules will not receive a
response}. When you contact your instructors, you are expected to be
professional in your communication. This includes:

\begin{itemize}
\tightlist
\item
  Providing a relevant description or statement in the email subject
  line. Do not leave the subject line blank or simply write, ``Hi.''
\item
  Providing your full name, uNID, and class section in the message.
\item
  Using appropriate salutations (e.g., Dr.~or Prof.~Yeo; recipient's
  name, if appropriate).
\item
  Using paragraphs, not just long blocks of text.
\item
  Proofreading your writing.
\item
  Providing a clear description of your problem and all relevant
  information.
\item
  Being polite in your emails. For example, you should end your messages
  with a signature, such as ``sincerely,'' ``regards,'' or ``thank
  you.''
\end{itemize}

\hypertarget{course-civility}{%
\subsection{Course Civility}\label{course-civility}}

Communication allows us to engage with others and broaden our
perspectives. How concepts are discussed, in the physical or virtual
classroom, is part of that process. Diverse perspectives and experiences
will inform and enhance our discussions. Each member of the class is
expected to foster a respectful, generous, and supportive environment
that makes room for productive difference and reasoned debate. Spirited
discussion is encouraged. However, incivility is a different story
entirely. Here is the basic etiquette that will be expected in the
course:

\begin{itemize}
\tightlist
\item
  Please address your classmates by name. There is a human being on the
  other side of the screen/room who also has struggles, doubts, and bad
  days.
\item
  \textbf{Civil} disagreement is encouraged! Approach differences in a
  manner that seeks clarity and better understanding by asking
  productive questions and by providing counterarguments that are
  supported with evidence.
\item
  Anytime you have a strong emotional reaction to something, pause
  before responding. Always seek to provide an argument that is
  supported by credible evidence based on the concepts discussed in this
  course.
\end{itemize}

\hypertarget{academic-misconduct}{%
\subsection{Academic Misconduct}\label{academic-misconduct}}

\begin{tcolorbox}[enhanced jigsaw, bottomrule=.15mm, leftrule=.75mm, toptitle=1mm, titlerule=0mm, breakable, toprule=.15mm, opacityback=0, colbacktitle=quarto-callout-warning-color!10!white, colback=white, title=\textcolor{quarto-callout-warning-color}{\faExclamationTriangle}\hspace{0.5em}{Warning}, bottomtitle=1mm, arc=.35mm, rightrule=.15mm, coltitle=black, colframe=quarto-callout-warning-color-frame, left=2mm, opacitybacktitle=0.6]

Academic misconduct will be punished to the fullest extent possible.
Anyone found guilty of academic misconduct should expect to fail this
course.

\end{tcolorbox}

It is expected that students comply with University of Utah policies
regarding academic honesty, including but not limited to refraining from
cheating, plagiarizing, misrepresenting one's work, and/or
inappropriately collaborating. This includes the use of generative
artificial intelligence (AI) tools without citation, documentation, or
authorization. Students are expected to adhere to the prescribed
professional and ethical standards of the profession/discipline for
which they are preparing. Any student who engages in academic dishonesty
or who violates the professional and ethical standards for their
profession/discipline may be subject to academic sanctions as per the
University of Utah's Student Code: Policy 6-410: Student Academic
Performance, Academic Conduct, and Professional and Ethical Conduct.

Plagiarism and cheating are serious offenses and may be punished by
failure on an individual assignment, and/or failure in the course.
Academic misconduct, according to the University of Utah Student Code:

\begin{quote}
``\ldots{} includes, but is not limited to, cheating, misrepresenting
one's work, inappropriately collaborating, plagiarism, and fabrication
or falsification of information\ldots It also includes facilitating
academic misconduct by intentionally helping or attempting to help
another to commit an act of academic misconduct.''
\end{quote}

For details on plagiarism and other important course conduct issues, see
the U's \href{http://regulations.utah.edu/academics/6-400.php}{Code of
Student Rights and Responsibilities}.

\hypertarget{curriculum-accommodations}{%
\subsection{Curriculum Accommodations}\label{curriculum-accommodations}}

Curriculum accommodations take two forms--scheduling and content
accommodations. On a case-by-case basis, if you submit the appropriate
documentation in advance of the conflict (when possible), scheduling
accommodations for assignments may be considered.

If you anticipate a scheduling conflict, please speak with me as soon as
possible. Without exception, it is your responsibility to plan for any
scheduling conflict.

\textbf{There will be no content accommodations in this course}. The
material has been selected for its pedagogical value in relation to the
concepts we are engaging. It is your responsibility to review the course
materials to be sure that this is a course you wish to take. More
information on the University's accommodation policy can be found in
\href{https://regulations.utah.edu/academics/6-100.php}{Policy 6-100}.

\hypertarget{emergency-plan}{%
\subsection{Emergency Plan}\label{emergency-plan}}

In the event of a University-wide emergency which prevents face-to-face
meetings, students should continue to stay current with our schedule as
posted in this syllabus and to attend to the course website on Canvas.
Information about the status of assignments and other course work due
during this period will be addressed on Canvas and, if necessary, by way
of email.

\hypertarget{sec-Upolicies}{%
\section{University Policies}\label{sec-Upolicies}}

\hypertarget{ada}{%
\subsection{ADA}\label{ada}}

The University of Utah seeks to provide equal access to its programs,
services, and activities for people with disabilities. All written
information in this course can be made available in an alternative
format with prior notification to the Center for Disability \& Access
(CDA). CDA will work with you and the instructor to make arrangements
for accommodations. Prior notice is appreciated. To read the full
accommodations policy for the University of Utah, please see Section Q
of the Instruction \& Evaluation regulations. In compliance with ADA
requirements, some students may need to record course content. Any
recordings of course content are for personal use only, should not be
shared, and should never be made publicly available. In addition,
recordings \textbf{must} be destroyed at the conclusion of the course.
If you will need accommodations in this class, or for more information
about what support they provide, contact the
\href{https://disability.utah.edu/}{Center for Disability \& Access}.

\hypertarget{safety}{%
\subsection{Safety}\label{safety}}

\textbf{Safety at the U.} The University of Utah values the safety of
all campus community members. You will receive important emergency
alerts and safety messages regarding campus safety via text message. For
more safety information and to view available training resources,
including helpful videos, visit safeu.utah.edu.

\textbf{Addressing Sexual Misconduct.} Title IX makes it clear that
violence and harassment based on sex and gender (includes sexual
orientation and gender identity/expression) is a civil rights offense
subject to the same kinds of accountability and the same kinds of
support applied to offenses against other protected categories such as
race, national origin, color, religion, age, status as a person with a
disability, veteran's status or genetic information. If you or someone
you know has been harassed or assaulted, you are encouraged to report it
to the \href{https://oeo.utah.edu/}{Title IX Coordinator in the Office
of Equal Opportunity and Affirmative Action} or the
\href{https://deanofstudents.utah.edu/}{Office of the Dean of Students}.
To report to the police, contact \href{https://dps.utah.edu/}{Campus
Police}. If you do not feel comfortable reporting to authorities, the
U's Victim-Survivor Advocates provide free, confidential, and
trauma-informed support services to students, faculty, and staff who
have experienced interpersonal violence. To privately explore options
and resources available to you with an advocate, contact the
\href{http://wellness.utah.edu/}{Center for Student Wellness}.

\newpage

\hypertarget{course-schedule}{%
\section{Course Schedule}\label{course-schedule}}

The schedule is tentative. Any changes will be announced on Canvas. Your
continued enrollment in this course constitutes an agreement to abide by
the policies and procedures in this syllabus.

\hypertarget{friday-20-jun-day-1-paradigmssome-early-communication-models}{%
\subsection{Friday (20-Jun; Day 1): Paradigms/Some Early Communication
Models}\label{friday-20-jun-day-1-paradigmssome-early-communication-models}}

Readings:

\begin{itemize}
\tightlist
\item
  Course syllabus
\item
  Schwartz, M. A. (2008). The importance of stupidity in scientific
  research. \emph{Journal of Cell Science}, \emph{121}(11), 1771--1771.
  https://doi.org/10.1242/jcs.033340
\end{itemize}

\begin{center}\rule{0.5\linewidth}{0.5pt}\end{center}

\hypertarget{monday-23-jun-day-2-the-nature-of-theory}{%
\subsection{Monday (23-Jun; Day 2): The Nature of
Theory}\label{monday-23-jun-day-2-the-nature-of-theory}}

Readings:

\begin{itemize}
\tightlist
\item
  Hornik, R., \& Yanovitzky, I. (2003). Using theory to design
  evaluations of communication campaigns: The case of the National Youth
  Anti-Drug Media Campaign. \emph{Communication Theory}, \emph{13}(2),
  204--224. https://doi.org/10.1111/j.1468-2885.2003.tb00289.x
\end{itemize}

Case Study Presentation:

\begin{itemize}
\tightlist
\item
  sign up here
\end{itemize}

\begin{center}\rule{0.5\linewidth}{0.5pt}\end{center}

\hypertarget{tuesday-24-jun-day-3-perceptions-of-risk}{%
\subsection{Tuesday (24-Jun; Day 3): Perceptions of
Risk}\label{tuesday-24-jun-day-3-perceptions-of-risk}}

Readings:

\begin{itemize}
\tightlist
\item
  Slovic, P., Fischhoff, B., \& Lichtenstein, S. (1982). Why study risk
  perception? \emph{Risk Analysis}, \emph{2}(2), 83--93.
  https://doi.org/10.1111/j.1539-6924.1982.tb01369.x
\end{itemize}

Case Study Presentation:

\begin{itemize}
\tightlist
\item
  sign up here
\end{itemize}

\begin{center}\rule{0.5\linewidth}{0.5pt}\end{center}

\hypertarget{wednesday-25-jun-day-4-knowledge-and-risk-perceptions}{%
\subsection{Wednesday (25-Jun; Day 4): Knowledge and Risk
Perceptions}\label{wednesday-25-jun-day-4-knowledge-and-risk-perceptions}}

Readings:

\begin{itemize}
\tightlist
\item
  Simis, M. J., Madden, H., Cacciatore, M. A., \& Yeo, S. K. (2016). The
  lure of rationality: Why does the deficit model persist in science
  communication? \emph{Public Understanding of Science}, \emph{25}(4),
  400--414. https://doi.org/10.1177/0963662516629749
\end{itemize}

Case Study Presentation:

\begin{itemize}
\tightlist
\item
  sign up here
\end{itemize}

\begin{center}\rule{0.5\linewidth}{0.5pt}\end{center}

\hypertarget{thursday-26-jun-day-5-media-coverage-of-risks}{%
\subsection{Thursday (26-Jun; Day 5): Media Coverage of
Risks}\label{thursday-26-jun-day-5-media-coverage-of-risks}}

Readings:

\begin{itemize}
\tightlist
\item
  Mayeda, A. M., Boyd, A. D., Paveglio, T. B., \& Flint, C. G. (2018).
  Media representations of water issues as health risks.
  \emph{Environmental Communication}, \emph{13}(7), 926--942.
  https://doi.org/10.1080/17524032.2018.1513054
\end{itemize}

Case Study Presentation:

\begin{itemize}
\tightlist
\item
  sign up here
\end{itemize}

\begin{center}\rule{0.5\linewidth}{0.5pt}\end{center}

\hypertarget{friday-27-jun-day-6-framing}{%
\subsection{Friday (27-Jun; Day 6):
Framing}\label{friday-27-jun-day-6-framing}}

Readings:

\begin{itemize}
\tightlist
\item
  Tversky, A., \& Kahneman, D. (1981). The framing of decisions and the
  psychology of choice. \emph{Science}, \emph{211}(4481), 453.
  https://doi.org/10.1126/science.7455683
\end{itemize}

Case Study Presentation:

\begin{itemize}
\tightlist
\item
  sign up here
\end{itemize}

\begin{center}\rule{0.5\linewidth}{0.5pt}\end{center}

\hypertarget{monday-30-jun-day-7-framing-agenda-setting-and-priming}{%
\subsection{Monday (30-Jun; Day 7): Framing, Agenda Setting, and
Priming}\label{monday-30-jun-day-7-framing-agenda-setting-and-priming}}

Readings:

\begin{itemize}
\tightlist
\item
  Scheufele, D. A., \& Tewksbury, D. (2007). Framing, agenda setting,
  and priming: The evolution of three media effects models.
  \emph{Journal of Communication}, \emph{57}(1), 9--20.
  https://doi.org/10.1111/J.1460-2466.2006.00326.X
\end{itemize}

Case Study Presentation:

\begin{itemize}
\tightlist
\item
  sign up here
\end{itemize}

\begin{center}\rule{0.5\linewidth}{0.5pt}\end{center}

\hypertarget{tuesday-1-jul-day-8-models-of-information-processing-elm-hsm}{%
\subsection{Tuesday (1-Jul; Day 8): Models of Information Processing
(ELM,
HSM)}\label{tuesday-1-jul-day-8-models-of-information-processing-elm-hsm}}

Readings:

\begin{itemize}
\tightlist
\item
  Luttrell, A. (2018). \emph{Dual Process Models of Persuasion}. In
  Oxford Research Encyclopedia of Psychology.
  https://doi.org/10.1093/acrefore/9780190236557.013.319
\end{itemize}

Case Study Presentation:

\begin{itemize}
\tightlist
\item
  sign up here
\end{itemize}

\begin{center}\rule{0.5\linewidth}{0.5pt}\end{center}

\hypertarget{wednesday-2-jul-day-9-tailoring-and-targeting}{%
\subsection{Wednesday (2-Jul; Day 9): Tailoring and
Targeting}\label{wednesday-2-jul-day-9-tailoring-and-targeting}}

Readings:

\begin{itemize}
\tightlist
\item
  Rimer, B. K., \& Kreuter, M. W. (2006). Advancing tailored health
  communication: A persuasion and message effects perspective.
  \emph{Journal of Communication}, \emph{56}(suppl\_1), S184--S201.
  https://doi.org/10.1111/j.1460-2466.2006.00289.x
\end{itemize}

Case Study Presentation:

\begin{itemize}
\tightlist
\item
  sign up here
\end{itemize}

\begin{center}\rule{0.5\linewidth}{0.5pt}\end{center}

\hypertarget{thursday-3-jul-day-10-health-belief-model}{%
\subsection{Thursday (3-Jul; Day 10): Health Belief
Model}\label{thursday-3-jul-day-10-health-belief-model}}

Readings:

\begin{itemize}
\tightlist
\item
  Janz, N. K., \& Becker, M. H. (1984). The Health Belief Model: A
  decade later. \emph{Health Education Quarterly}, \emph{11}(1), 1--47.
  https://doi.org/10.1177/109019818401100101
\end{itemize}

Case Study Presentation:

\begin{itemize}
\tightlist
\item
  sign up here
\end{itemize}

\begin{center}\rule{0.5\linewidth}{0.5pt}\end{center}

\hypertarget{friday-4-jul-day-11-social-amplification-of-risk}{%
\subsection{Friday (4-Jul; Day 11): Social Amplification of
Risk}\label{friday-4-jul-day-11-social-amplification-of-risk}}

Readings:

\begin{itemize}
\tightlist
\item
  Kasperson, R. E., Webler, T., Ram, B., \& Sutton, J. (2022). The
  social amplification of risk framework: New perspectives. \emph{Risk
  Analysis}, \emph{42}(7), 1367--1380.
  https://doi.org/10.1111/risa.13926
\end{itemize}

Case Study Presentation:

\begin{itemize}
\tightlist
\item
  sign up here
\end{itemize}

\begin{center}\rule{0.5\linewidth}{0.5pt}\end{center}

\hypertarget{monday-7-jul-day-12-health-communication-campaigns-social-media}{%
\subsection{Monday (7-Jul; Day 12): Health Communication Campaigns \&
Social
Media}\label{monday-7-jul-day-12-health-communication-campaigns-social-media}}

Readings:

\begin{itemize}
\tightlist
\item
  McGurk, M. D., Ogawa, G., Inoue, K., Wills, C., Ching, L. K., Shalaby,
  A. K., Kong, N., Hansen Smith, H., Lee, J., Irvin, L., \& Keliikoa, L.
  B. (2025). Sweet Lies! Lessons learned from Hawai`i's sweetened fruit
  drink countermarketing campaign. \emph{Journal of Health
  Communication}, \emph{30}(sup1), 14--27.
  https://doi.org/10.1080/10810730.2025.2461588
\end{itemize}

Case Study Presentation:

\begin{itemize}
\tightlist
\item
  sign up here
\end{itemize}

\begin{center}\rule{0.5\linewidth}{0.5pt}\end{center}

\hypertarget{tuesday-8-jul-day-13-making-health-communication-programs-work}{%
\subsection{Tuesday (8-Jul; Day 13): Making Health Communication
Programs
Work}\label{tuesday-8-jul-day-13-making-health-communication-programs-work}}

Readings:

\begin{itemize}
\tightlist
\item
  National Cancer Institute. (2001). \emph{Making Health Communication
  Programs Work}. U.S. Department of Health \& Human Services.
  https://stacks.cdc.gov/view/cdc/24017
\end{itemize}

Case Study Presentation:

\begin{itemize}
\tightlist
\item
  sign up here
\end{itemize}

\begin{center}\rule{0.5\linewidth}{0.5pt}\end{center}

\hypertarget{wednesday-9-jul-day-14-thursday-10-jul-day-15-group-project-presentations}{%
\subsection{Wednesday (9-Jul; Day 14) \& Thursday (10-Jul; Day 15):
Group Project
Presentations}\label{wednesday-9-jul-day-14-thursday-10-jul-day-15-group-project-presentations}}



\end{document}
