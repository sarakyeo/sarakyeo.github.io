% Options for packages loaded elsewhere
\PassOptionsToPackage{unicode}{hyperref}
\PassOptionsToPackage{hyphens}{url}
\PassOptionsToPackage{dvipsnames,svgnames,x11names}{xcolor}
%
\documentclass[
  letterpaper,
  DIV=11,
  numbers=noendperiod]{scrartcl}

\usepackage{amsmath,amssymb}
\usepackage{lmodern}
\usepackage{iftex}
\ifPDFTeX
  \usepackage[T1]{fontenc}
  \usepackage[utf8]{inputenc}
  \usepackage{textcomp} % provide euro and other symbols
\else % if luatex or xetex
  \usepackage{unicode-math}
  \defaultfontfeatures{Scale=MatchLowercase}
  \defaultfontfeatures[\rmfamily]{Ligatures=TeX,Scale=1}
  \setmainfont[]{Latin Modern Roman}
  \setsansfont[]{Latin Modern Roman}
\fi
% Use upquote if available, for straight quotes in verbatim environments
\IfFileExists{upquote.sty}{\usepackage{upquote}}{}
\IfFileExists{microtype.sty}{% use microtype if available
  \usepackage[]{microtype}
  \UseMicrotypeSet[protrusion]{basicmath} % disable protrusion for tt fonts
}{}
\makeatletter
\@ifundefined{KOMAClassName}{% if non-KOMA class
  \IfFileExists{parskip.sty}{%
    \usepackage{parskip}
  }{% else
    \setlength{\parindent}{0pt}
    \setlength{\parskip}{6pt plus 2pt minus 1pt}}
}{% if KOMA class
  \KOMAoptions{parskip=half}}
\makeatother
\usepackage{xcolor}
\setlength{\emergencystretch}{3em} % prevent overfull lines
\setcounter{secnumdepth}{-\maxdimen} % remove section numbering
% Make \paragraph and \subparagraph free-standing
\ifx\paragraph\undefined\else
  \let\oldparagraph\paragraph
  \renewcommand{\paragraph}[1]{\oldparagraph{#1}\mbox{}}
\fi
\ifx\subparagraph\undefined\else
  \let\oldsubparagraph\subparagraph
  \renewcommand{\subparagraph}[1]{\oldsubparagraph{#1}\mbox{}}
\fi


\providecommand{\tightlist}{%
  \setlength{\itemsep}{0pt}\setlength{\parskip}{0pt}}\usepackage{longtable,booktabs,array}
\usepackage{calc} % for calculating minipage widths
% Correct order of tables after \paragraph or \subparagraph
\usepackage{etoolbox}
\makeatletter
\patchcmd\longtable{\par}{\if@noskipsec\mbox{}\fi\par}{}{}
\makeatother
% Allow footnotes in longtable head/foot
\IfFileExists{footnotehyper.sty}{\usepackage{footnotehyper}}{\usepackage{footnote}}
\makesavenoteenv{longtable}
\usepackage{graphicx}
\makeatletter
\def\maxwidth{\ifdim\Gin@nat@width>\linewidth\linewidth\else\Gin@nat@width\fi}
\def\maxheight{\ifdim\Gin@nat@height>\textheight\textheight\else\Gin@nat@height\fi}
\makeatother
% Scale images if necessary, so that they will not overflow the page
% margins by default, and it is still possible to overwrite the defaults
% using explicit options in \includegraphics[width, height, ...]{}
\setkeys{Gin}{width=\maxwidth,height=\maxheight,keepaspectratio}
% Set default figure placement to htbp
\makeatletter
\def\fps@figure{htbp}
\makeatother

\KOMAoption{captions}{tableheading}
\makeatletter
\makeatother
\makeatletter
\makeatother
\makeatletter
\@ifpackageloaded{caption}{}{\usepackage{caption}}
\AtBeginDocument{%
\ifdefined\contentsname
  \renewcommand*\contentsname{Table of contents}
\else
  \newcommand\contentsname{Table of contents}
\fi
\ifdefined\listfigurename
  \renewcommand*\listfigurename{List of Figures}
\else
  \newcommand\listfigurename{List of Figures}
\fi
\ifdefined\listtablename
  \renewcommand*\listtablename{List of Tables}
\else
  \newcommand\listtablename{List of Tables}
\fi
\ifdefined\figurename
  \renewcommand*\figurename{Figure}
\else
  \newcommand\figurename{Figure}
\fi
\ifdefined\tablename
  \renewcommand*\tablename{Table}
\else
  \newcommand\tablename{Table}
\fi
}
\@ifpackageloaded{float}{}{\usepackage{float}}
\floatstyle{ruled}
\@ifundefined{c@chapter}{\newfloat{codelisting}{h}{lop}}{\newfloat{codelisting}{h}{lop}[chapter]}
\floatname{codelisting}{Listing}
\newcommand*\listoflistings{\listof{codelisting}{List of Listings}}
\makeatother
\makeatletter
\@ifpackageloaded{caption}{}{\usepackage{caption}}
\@ifpackageloaded{subcaption}{}{\usepackage{subcaption}}
\makeatother
\makeatletter
\@ifpackageloaded{tcolorbox}{}{\usepackage[many]{tcolorbox}}
\makeatother
\makeatletter
\@ifundefined{shadecolor}{\definecolor{shadecolor}{rgb}{.97, .97, .97}}
\makeatother
\makeatletter
\makeatother
\ifLuaTeX
  \usepackage{selnolig}  % disable illegal ligatures
\fi
\IfFileExists{bookmark.sty}{\usepackage{bookmark}}{\usepackage{hyperref}}
\IfFileExists{xurl.sty}{\usepackage{xurl}}{} % add URL line breaks if available
\urlstyle{same} % disable monospaced font for URLs
\hypersetup{
  colorlinks=true,
  linkcolor={blue},
  filecolor={Maroon},
  citecolor={Blue},
  urlcolor={Blue},
  pdfcreator={LaTeX via pandoc}}

\title{COMM 7370\\
Quantitative Communication Research\\
\vspace{0.5cm} \large Wednesdays (2:00 - 4:50 pm)\\
LNCO 2630\\
\vspace{0.5cm} Spring 2023\\
The University of Utah}
\author{Sara K. Yeo\\
\href{mailto:sara.yeo@utah.edu}{\nolinkurl{sara.yeo@utah.edu}}}
\date{}

\begin{document}
\maketitle
\ifdefined\Shaded\renewenvironment{Shaded}{\begin{tcolorbox}[breakable, borderline west={3pt}{0pt}{shadecolor}, interior hidden, enhanced, sharp corners, frame hidden, boxrule=0pt]}{\end{tcolorbox}}\fi

This graduate seminar is an introductory course in quantitative research
for communication-related topics. We will examine how research questions
are developed into a research project. Additionally, we will learn how
to select appropriate research techniques, measure concepts, draw
samples, interpret results, and communicate our research.

Key topics include:

\begin{itemize}
\item
  Formalizing hypotheses and research questions grounded in theory
\item
  Testing hypotheses and research questions
\item
  Conceptual and operational definitions
\item
  Measurement, sampling, and research design
\item
  Data analysis in communication research
\end{itemize}

The main objectives of this course are:

\begin{enumerate}
\def\labelenumi{\arabic{enumi}.}
\item
  To offer a theoretical perspective on quantitative social science
  research with a focus on surveys and experiments in communication.
\item
  To familiarize you with data analysis using two software packages, R
  and IBM SPSS Statistics.
\item
  To stimulate ideas for original research and help you conduct data
  analysis for your own future research projects.
\item
  To generate a class study and paper using quantitative research for
  presentation at a conference and publication in a journal.
\end{enumerate}

It is challenging to grasp research methods without doing research. As a
result, much of the course will be spent linking concepts to survey
questions, collecting and analyzing data, trying to make sense of
output, and linking data analysis to research questions and hypotheses.
We will also have working labs at the end of the semester during which
we will work on our class manuscript.

\hypertarget{required-readings}{%
\section{Required Readings}\label{required-readings}}

For most weeks, I have compiled a non-exhaustive set of readings. I will
post the weekly reading list on Canvas. If PDFs are not available
through the Marriott Library resources, the files will be posted on
Canvas.

You are expected to complete these readings before the start of each
meeting. These readings are intended as a point-of-entry into the
week\textbackslash's content. They may also be useful if you are trying
to compile a reading list for your preliminary examinations.

\hypertarget{computer-software}{%
\section{Computer Software}\label{computer-software}}

You will need Microsoft Office (Word, Excel, PowerPoint) and software
for data analysis (\href{https://www.r-project.org/}{R},
\href{https://onthehub.com/products/a4db50af-41be-eb11-813b-000d3af41938}{IBM
SPSS Statistics}). You will have to either purchase IBM SPSS Statistics
or use it through the remote software tools via the
\href{https://lib.utah.edu/services/knowledge-commons/remote-software/}{Marriott
Library}. Please note that you do not have to have them ready for the
first day of class. We will set these up during class when needed.

Additionally, access to a text-editor (e.g., Wordpad, TextEdit,
Notepad++) and Adobe Acrobat (free for UofU students) is recommended.

You will need access to \href{https://utah.instructure.com/}{Canvas}. I
expect you to check the course website on Canvas regularly.
Announcements, assignments, readings, discussions, etc., will be posted
there. You should be familiar with and comfortable using Canvas and
Zoom. If you need help with Canvas, visit the
\href{https://community.canvaslms.com/docs/DOC-10701}{Canvas Getting
Started Guide for Students}.

\hypertarget{course-requirements}{%
\section{Course Requirements}\label{course-requirements}}

Your grade in this course will be based on the following:

\begin{itemize}
\item
  Discussion leadership (40\%)
\item
  Lab assignments (30\%)
\item
  Participation and contribution to class paper (30\%)
\end{itemize}



\end{document}
